\documentclass{scrreprt}
\usepackage{listings}
\usepackage{underscore}
\usepackage[francais]{babel}
\usepackage[utf8]{inputenc}
\usepackage[bookmarks=true]{hyperref}
\usepackage{tabularx}

\hypersetup{
    bookmarks=false,    % show bookmarks bar?
    pdftitle={Document de spécification des exigences du logiciel}
pdfauthor={Jean-Philippe Caissy\\Stéphan Michaud\\Guillaume Auger\\Maxime
Bélanger\\Justin Michaud-Ouelette\\Frédéric Sevillano}
%pdfsubject={TeX and LaTeX},                        % subject of the document    %pdfkeywords={TeX, LaTeX, graphics, images}, % list of keywords
    colorlinks=true,       % false: boxed links; true: colored links
    linkcolor=blue,       % color of internal links
    citecolor=black,       % color of links to bibliography
    filecolor=black,        % color of file links
    urlcolor=purple,        % color of external links
    linktoc=page            % only page is linked
}%
\def\myversion{1.0 }
\title{%
\flushright
\rule{16cm}{5pt}\vskip1cm
\Huge{Document de spécification des exigences du logiciel}\\
\vspace{2cm}
pour\\
\vspace{2cm}
Nitcorn\\
\vspace{2cm}
%\LARGE{Release 1.0\\}
%\vspace{2cm}
%\LARGE{Version \myversion approved\\}
%\vspace{2cm}
Préparé par Jean-Philippe Caissy\\Stéphan Michaud\\Guillaume Auger\\Maxime
Bélanger\\Justin Michaud-Ouelette\\Frédéric Sevillano\\
\vfill
\rule{16cm}{5pt}
}
\date{}
\usepackage{hyperref}
\begin{document}
\maketitle
\tableofcontents
\chapter*{Historique de révision}
\begin{tabularx}{\textwidth}{|r|X|r|}
    \hline
    Nom & Description & Date \\
    \hline
    Stéphan Michaud & Création & 2013/02/04 \\
    \hline
    Équipe & Ébauche & 2013/02/26 \\
    \hline
    Jean-Philippe Caissy & Convertion en \LaTeX & 2013/03/04 \\
    \hline
\end{tabularx}

\chapter{Introduction}
Ce chapitre permet d'introduire la documentation du serveur Internet Nitcorn. Ilpermettra d'éliminer toute ambiguïté concernant la documentation
de notre application à l'aide de définitions, acronymes et
abréviations. Les objectifs et les portées seront décrites pour
avoir une meilleure compréhension du document.
\section{Objet}
L'objectif de ce document est de présenter une description
détaillée des différents modules de l'application Nitcorn. De plus,
une spécification d'interface externe sera décrite permettant aux applications
Nit existantes
d'utiliser un serveur web.\\
Ce document s'adresse aux développeur du projet Nitcorn ainsi qu'aux potentiels
contributeurs externes.
\section{Portée}
L'application dont le présent document décrit est Nitcorn. Les différentes
composantes seront :
\begin{itemize}
    \item Écoute sur un socket;
    \item Réception de requête HTTP 1.0 et 1.1;
    \item Traitement et réponse à une requête HTTP;
    \item Exposer une API pour gérer la configuration;
    \item Gestion de log;
    \item Transmettre les requêtes vers un serveur mandataire; et
    \item Permettre la ré-écriture des requêtes HTTP.
\end{itemize}
Les éléments suivants ne sont pas traités par Nitcorn. Ainsi une couche
d'abstraction
sera développé pour utiliser les librairies externes avec l'interface native de
Nit.

\begin{itemize}
    \item Chiffrement et déchiffrement (SSL/TLS);
    \item Boucle évènementiel gèrant les entrées et sorties bloquantes; et
    \item Persistance de la configuration;
\end{itemize}

Les 3 éléments ci-hauts ne seront pas développés en Nit puisqu'il existe des
librairies déjà existante matures et testés en production permettant de répondreà ses demandes.
\\
De plus, nous avons décidé d'utiliser un patron de conception basée sur
une boucle d'évènements pour les communications bloquantes sur les sockets. Celapermet d'utiliser qu'un seul processus pour le serveur, tout en restant très
performant.
Ce mécanisme permet d'utiliser des mécanismes existant du système d'exploitation
pour
gérer de manière asynchrone les entrées et sorties\cite{c10k}.

\section{Définitions, acroynmes et abréviations}
Voici les définitions des différents termes techniques utilisés dans ce
document.
\begin{description}
\item[Protocole HTTP] Protocole de communication applicative d'architecture
client-serveur\cite{http}.
\item[HTTP 1.0] Version 1.0 du protocole HTTP ne permettant pas d'avoir des
connexion keep-alive.
\item[HTTP 1.1] Version 1.1 du protocole HTTP permettant entre autre d'avoir des
connexion keep-alive\cite{http1.0}.
\item[Nit] Langage de programmation open source orienté objet développé par le
groupe de recherche sur l'étude, la spécification et l'implémentation des
langages informatique du département d'informatique à l'Université du Québec à
Montréal.
    \item[Boucle d'évènement asynchrone] 
    \item[Fil d'exécution] Traitement d'une suite d'instructions par la machine sur un programme donné.
    \item[MIME]
    \item[Méthode HTTP]
    \item[Cookie]
    \item[URL]
    \item[Apache]
    \item[Nginx]
    \item[Authentification Basic]
    \item[Authentification Digest]
    \item[Fastcgi]
    \item[Socket] Interface logiciel de connexion réseau.
    \item[File descriptor] identifiant d'un fichié ouvert par le système d'exploitation
    \item[x86] Architecture de processeur 32 bits développée par Intel.
    \item[x86_64] Architecture de processeur 64 bit développée par AMD, mais comptabile avec la famille de processeur 64 bit d'Intel (à l'exception de Itanimum).
\end{description}

\section{Références}
\begin{tabularx}{\textwidth}{|l|X|l|}
    \hline
    Ref. & Numéro du document & Titre \\
    \hline
    \cite{ieefr} & IEEE 830-1993 & Norme IEEE 830-1993 \\
    \hline
    \cite{http} & RFC2616 & Hypertext Transfer Protocol -- HTTP/1.1 \\
    \hline
\end{tabularx}

\section{Vue d'ensemble}
@TODO
\chapter{Description générale}
\section{Environnement}
Le but premier de Nitcorn est d'offrir aux applications développés en Nit
d'utiliser
un serveur web. Ainsi, il est important d'offrir des fonctionnalités de serveur
au langage afin qu'il puisse être placée derrière un serveur web (Nginx, Apache,
\ldots).
Sous cette configuration, il est quand même essentiel de pouvoir répondre à des
requêtes web, mais des fonctionnalités tel que la gestion des logs ou la
persistance
des configuration n'est plus inutile car ces fonctionnalités seraient gérés par
le serveur web.\\
Par contre, cela n'empêcherait pas Nitcorn d'être directement exposé à Internet
et agir en tant qu'un serveur web exactement comme c'est le cas pour les autres
serveur webs (Nginx, Apache, \ldots). Sous cette configuration, les
fonctionnalités
mentionnés ci-haut doivent être utilisés pour assurer les requis de traçabilité
et de fonctionnalités essentiels d'un serveur web.

\subsection{Interfaces avec le sysyème}
Nitcorn sera utilisé sur deux système : un serveur web et un environnement de développement.
Le serveur web sera le système ayant comme fonctionnalité d'agir en tant que serveur web, alors
que l'environnement de développement sera le système utilisé par les développeurs
pour développer des applications web en Nit.

\subsection{Interfaces avec les utilisateurs}
Comme il est disponible avec les autres serveurs web, un système de log est
disponible permettant à un utilisateur de déterminer la cause du problème. Sur
un autre point, le logiciel aura une API permettant aux applications Nit de 
communiquer sur le web sans avoir de se soucier des configurations. Cette
API sera grandement inspirée par l'interface universelle WGSI développé pour
le langage Python. Le logiciel Nitcorn offira aussi une interface pour les
applications Nit.

\subsection{Interfaces avec le matériel}
Nitcorn dépent du compilateur de Nit. Celui-ci est compatible pour le moment avec
les architectures x86 et x86_64.

\subsection{Interfaces avec les logiciels}
Nitcorn ne supporte que les systèmes d'exploitation de style UNIX (GNU/Linux, *BSD, Mac OS X, etc).
De plus, Nitcorn a besoin des logiciels suivants lors de la compilation : \\
\\
\begin{tabular}{|l|l|l|l|l|}
    \hline
    Nom & Mnémonique & Spécification & Version & Source \\
    \hline
    Nitc & N/A & FFI & N/A & \url{https://github.com/xymus/nit/tree/ffi} \\
    \hline
    Libevent & N/A & N/A & 2.0.X & \url{http://libevent.org/} \\
    \hline
    SQLite & N/A & N/A & 3.X & \url{https://www.sqlite.org/} \\
    \hline

\end{tabular}

\section{Fonctions}
Les différentes fonctions sont séparaés en deux groupes : serveur et application Nit.
Le serveur est Nitcorn, alors que l'application Nit est l'application qui utilise
Nitcorn pour répondre à des requêtes web dynamiquement.

\subsection{Serveur}
\subsubsection{Réception d'une requête}

\subsubsection{Traiter une requête}

\subsubsection{Lire la configuration}

\subsubsection{Écrire la configuration}

\subsubsection{Enregistrer les évènements}

\subsection{Application Nit}
\subsubsection{Recevoir une requête HTTP}

\subsubsection{Traiter une requête HTTP}

\section{Caractéristiques des utilisateurs}

\section{Contraintes}

\subsection{Mémoire}
Nous n'imposons aucune contrainte mémoire. Par contre, la quantité de \textbf{file descriptor} ouvert en même temps sera limité par le noyau du système d'exploitation. La limite est à la discrétion de l'utilisateur. Le serveur devra gerer l'atteinte de cette limite.
\subsection{Exécution}
Le traitement des requêtes dois s'effectuer de façon asynchrone sur un seul fil d'exécution. Plusieurs requêtes doivent être traitables simultané sans bloquer les nouvelles requêtes.


\section{Hypothèse et dépendances}
\chapter{Exigences spécifiques}

\begin{thebibliography}{9}
\bibitem{ieefr}
  Société d'informatique IEEE,
\emph{Norme IEEE 830-1993, Pratique recommandée par IEEE pour la préparation de
spécifications d’exigences de logiciel}, 1993
\url{http://www.cours.polymtl.ca/log3410/bibliographie/IEEE/Pratique_Recommandee_Par_IEEE_pour_la%20_Specification.pdf}.
\bibitem{c10k}
  Dan Kegel
\emph{The C10K Problem}, 1999, \url{http://www.kegel.com/c10k.html#strategies}\bibitem{http}
    The Internet Society,
\emph{Hypertext Transfer Protocol -- HTTP/1.1}, 1999,
\url{https://tools.ietf.org/html/rfc2616}
\bibitem{http1.0}
    The Internet Society,
\emph{Hypertext Transfer Protocol -- HTTP/1.0}, 1996,
\url{https://tools.ietf.org/html/rfc1945}
\end{thebibliography}


% add other chapters and sections to suit
\end{document}
