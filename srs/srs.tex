\documentclass{scrreprt}
\usepackage{listings}
\usepackage{underscore}
\usepackage[francais]{babel}
\usepackage[utf8]{inputenc}
\usepackage[bookmarks=true]{hyperref}
\usepackage{tabularx}
\hypersetup{
    bookmarks=false,    % show bookmarks bar?
    pdftitle={Document de spécification des exigences du logiciel}
    pdfauthor={Jean-Philippe Caissy\\Stéphan Michaud\\Guillaume Auger\\Maxime Bélanger\\Justin Michaud-Ouelette\\Frédéric Sevillano}
    %pdfsubject={TeX and LaTeX},                        % subject of the document
    %pdfkeywords={TeX, LaTeX, graphics, images}, % list of keywords
    colorlinks=true,       % false: boxed links; true: colored links
    linkcolor=blue,       % color of internal links
    citecolor=black,       % color of links to bibliography
    filecolor=black,        % color of file links
    urlcolor=purple,        % color of external links
    linktoc=page            % only page is linked
}%
\def\myversion{1.0 }
\title{%
\flushright
\rule{16cm}{5pt}\vskip1cm
\Huge{Document de spécification des exigences du logiciel}\\
\vspace{2cm}
pour\\
\vspace{2cm}
Nitcorn\\
\vspace{2cm}
%\LARGE{Release 1.0\\}
%\vspace{2cm}
%\LARGE{Version \myversion approved\\}
%\vspace{2cm}
Préparé par Jean-Philippe Caissy\\Stéphan Michaud\\Guillaume Auger\\Maxime Bélanger\\Justin Michaud-Ouelette\\Frédéric Sevillano\\
\vfill
\rule{16cm}{5pt}
}
\date{}
\usepackage{hyperref}
\begin{document}
\maketitle
\tableofcontents
\chapter*{Historique de révision}
\begin{tabularx}{\textwidth}{r|X|r}
    \hline
    Nom & Description & Date \\
    \hline
    Stéphan Michaud & Création & 2013/02/04 \\
    \hline
    Équipe & Ébauche & 2013/02/26 \\
    \hline
    Jean-Philippe Caissy & Convertion en \LaTeX & 2013/03/04 \\
    \hline
\end{tabularx}

\chapter{Introduction}
Ce chapitre permet d’introduire la documentation de Nitcorn. Il
permettra d’éliminer toute ambiguïté concernant la documentation
de notre application à l’aide de définitions, acronymes et
abréviations. Les objectifs et les portées seront décrites pour
avoir une meilleure compréhension du document.
\section{Objet}
L’objectif de ce document est de présenter une description
détaillée du système Nitcorn. Il explique les fonctionnalités et
les besoins de l’application serveur. Les spécifications de
l’application ne seront pas définies dans ce document.\\ \\ Ce
document s’adresse aux développeurs du projet Nitcorn et aux
potentiels utilisateurs.
\section{Portée}
Nitcorn est un serveur web :
\begin{itemize}
    \item Écoute des sockets;
    \item Réception de requête HTTP;
    \item Traitement et réponse à une requête HTTP;
    \item Gestion du contenu;
    \item Persistance de la configuration;
    \item Fichiers locaux de permission et d'authentification;
    \item Configuration via une API; et
    \item Protocole Nit de service web.
\end{itemize}

\section{Définitions}
\section{Acronymes}
\section{Abréviations}
\chapter{Description générale}
\section{Environnement}
\section{Fonctions}
\section{Caractéristiques des utilisateurs}
\section{Contraintes}
\section{Hypothèse et dépendances}
\chapter{Exigences spécifiques}
% add other chapters and sections to suit
\end{document}
